% Awesome Source CV LaTeX Template
%
% This template has been downloaded from:
% https://github.com/darwiin/awesome-neue-latex-cv
%
% Author:
% Christophe Roger
%
% Template license:
% CC BY-SA 4.0 (https://creativecommons.org/licenses/by-sa/4.0/)

%Section: Work Experience
\sectionTitle{Expériences Professionnelles}{\faSuitcase}
%\renewcommand{\labelitemi}{$\bullet$}
\begin{experiences}
  \experience
    {Aujourd'hui}   {Ingénieur électronique et système | OpenWork / Cowboy}{\link{https://www.openwork.co/fr/}{OpenWork}}{Lyon, France}
    {Avril 2021} {
                      Ingénierie électronique et système pour le développement d'un vélo électrique connecté (client \link{https://cowboy.com/}{Cowboy} en portage salarial à travers OpenWork)
                      \begin{itemize}
                        \item Conception de l'architecture électronique et système du vélo
                        \item Développement des modules électroniques (carte de connectivité et contrôle du vélo, chargeur sans-fil intégré, etc.)
                        \item Industrialisation des cartes électroniques : design-to-cost, design for manufacturing, sourcing de partenaires industriels (Europe, Asie), sourcing de composants, etc.
                        \item Développement d'algorithmes de commandes moteur
                      \end{itemize}
                    }
                    {IoT, Management, Hardware, Software}
                    
  \emptySeparator

  \experience
    {Avril 2021}   {Directeur technique | Développeur}{\link{https://rtone.fr/}{Rtone}}{Lyon, France}
    {Février 2017} {
                      Intrapreneuriat et développement hardware au sein du bureau d'étude IoT Rtone
                      \begin{itemize}
                        \item Développement de produits connectés B2B/B2C (WiFi, BLE, Sigfox, LoRa, etc.)
                        \item Industrialisation de produits connectés : design-to-cost, design for manufacturing, sourcing de partenaires industriels (Europe, Asie), sourcing de composants, etc.
                        \item Avant-vente de projets : accompagnement des commerciaux dans la construction et présentation d'offres
                        \item Gestion projet (Agile) : synchronisation des équipes, animation des travaux techniques, communication client, suivi de planning/budget
                        \item Management des équipes techniques hardware/software (35 personnes) : mise en place de méthodes de développements agiles, mise en place de process/workflow de développement/qualité, mise en place de base de connaissance interne, recrutement de profil technique, accompagnement de l'évolution des collaborateurs, liens entre les équipes commerciales et techniques, veille technique, etc.
                        \item Stratégie et construction d'une vision pour le futur de Rtone en collaboration avec le CEO, co-construction d'une culture d'entreprise
                      \end{itemize}
                    }
                    {IoT, Management, Hardware, Software}
                    
  \emptySeparator

  \experience
    {Octobre 2016}   {Développeur Hardware | Imprimantes 3D}{\link{https://formlabs.com/}{Formlabs}}{Boston, USA}
    {Avril 2016} {
                      Contribution au développement électronique de la Form 2, imprimante 3D SLA (Stéréolithographie)
                      \begin{itemize}
                        \item Design d’une carte de contrôle de galvanomètres : design d’une alimentation à découpage Ćuk, design d’un amplificateur de puissance 40 W, routage de PCB multicouches haute densité mixte
                        \item Évaluation de moteurs pas-à-pas et de contrôleurs (amélioration de la vitesse d’impression et réduction du bruit)
                        \item Design d’une carte pour la calibration mécanique en usine de la Form 2
                        \item Test et validation : test d’alimentations à découpage (stabilité, compensation, réponse aux transitoires de charges/tension, tenues en courant, CEM, etc.), test d’amplificateurs opérationnels/de puissance (stabilité, compensation, tenue en courant, bande passante, etc.), test d’endurance et de vieillissement
                      \end{itemize}
                    }
                    {Analogique, Numérique, Alimentations}
                    
  \emptySeparator
  
  \experience
    {Février 2015} {Développeur Hardware | Système d'infotainment connecté}{\link{https://www.parrot.com/}{Parrot}}{Paris, France}
    {Août 2014}    {
                      Contribution au développement électronique du système d’infotainment automobile connecté «RnB6»
                      «RnB6» a remporté deux Innovation Awards au CES 2015
                      \begin{itemize}
                        \item Développement de spécifications hardware (pin mapping, horloges, alimentations, etc.)
                        \item Design, routage et suivi de fabrication/assemblage de PCBs multicouches haute-fréquence rigides et flexibles                        
                        \item ITest et validation de circuits électroniques mixtes : bus de données (USB, I2C, UART), interfaces vidéo (MIPI-DSI, HDMI, MHL, CVBS) et filtres analogiques, test CEM pour environnement automobile
                      \end{itemize}
                    }
                    {Analogique, Numérique, Bus de données}
\end{experiences}
